%DEFINITIONS FOR THE PAPER
\documentclass[a4paper, 11pt]{article}
\usepackage[utf8]{inputenc}
\usepackage[T1]{fontenc}
\usepackage[top=3.0cm, bottom=2.43cm, right=3.0cm, left=3.0cm]{geometry}
%\usepackage{fontspec} % include OTF fonts, requires XeTeX
%\defaultfontfeatures{Ligatures=TeX}
\usepackage{filecontents}
\usepackage[natbib=true, style=numeric, backend=biber]{biblatex}
\addbibresource{references-esensing.bib}
%\setcitestyle{open={(},close={)}}
\usepackage{subfigure} % multiple figures in LaTeX
\usepackage{wrapfig}
%\usepackage{comment}
\usepackage{multirow} % multiplerows in tables
\setlength{\parindent}{0.25in}
\setlength{\parskip}{1.3em}
\renewcommand{\baselinestretch}{1.2}
\usepackage{titlesec}
\usepackage{booktabs} % Better horizontal rules in tables
\usepackage[all,defaultlines=3]{nowidow} % avoids widow and orphan sentences
\usepackage{amsmath} % good math formulas
\usepackage{relsize} % increase size of formulas
\usepackage{subfiles} % to handle multiple documents
%microtype package
\usepackage[activate={true,nocompatibility}, final, tracking=true,kerning=false,spacing=false]{microtype}
\usepackage{authblk}
\renewcommand\Authfont{\fontsize{11}{13}\selectfont}
\renewcommand\Affilfont{\fontsize{10}{12}\itshape}
\titleformat*{\section}{\large\bfseries}
\pagenumbering{gobble}% Remove page numbers (and reset to 1)

%END OF DEFINITIONS

% Paper title

\title{\bfseries Satellite image time series analysis with the sits package}
%no date
\date{\vspace{-5ex}}
\author[1]{\normalsize Gilberto Camara}
\author[2]{\normalsize Victor Maus}
\author[1]{\normalsize Luiz Fernando Assis}
\author[1]{\normalsize Rolf Simoes}
\author[1]{\normalsize Adeline Maciel}
\author[1]{\normalsize Gilberto Ribeiro de Queiroz}

\affil[1]{\small Image Processing Division, National Institute for Space Research (INPE), Av dos Astronautas 1758, Sao Jose dos Campos, 12227-001 Brazil. \break
	Email: \{gilberto.camara\}@inpe.br}
\affil[2]{\small International Institute for Applied System Analyis, Schlossplatz 2, 2631 Laxenburg, Austria}

\usepackage{Sweave}
\begin{document}
\Sconcordance{concordance:sits_intro.tex:sits_intro.Rnw:%
1 50 1 1 0 15 1 1 2 1 0 1 1 1 4 2 0 1 2 1 1 1 2 1 3 6 0 1 2 19 1}


\fontfamily{put}\selectfont % Utopia
\renewcommand*{\bibfont}{\small}
\maketitle

%1. Introduction:
%2. Time Series for LUCC Classification
%3. Land
%4. A framework for data modelling on land use semantics


A satellite image time series is...

\begin{Schunk}
\begin{Sinput}
> library(sits)
> URL <- "http://www.dpi.inpe.br/tws/wtss"
> sits_infoWTSS(URL)
\end{Sinput}
\begin{Soutput}
-----------------------------------------------------------
The WTSS server URL is http://www.dpi.inpe.br/tws/wtss
------------------------------------------------------------
Coverage: mod13q1_512
Description: Vegetation Indices 16-Day L3 Global 250m
Source: https://lpdaac.usgs.gov/dataset_discovery/modis/modis_products_table/mod13q1
Bands: 
  name                            description
1 ndvi                      250m 16 days NDVI
2  evi                       250m 16 days EVI
3  red  250m 16 days red reflectance (Band 1)
4  nir  250m 16 days NIR reflectance (Band 2)
5 blue 250m 16 days blue reflectance (Band 3)
6  mir  250m 16 days MIR reflectance (Band 7)

Spatial extent: (, ) - (, )
Time range:  to 
Temporal resolution:  days 
Spatial resolution:  metres 
----------------------------------------------------------------------------------
Coverage: itobi
Description: UNKNOWN
Source: UNKNOWN
Bands: 
   name description
1    b1     UNKNOWN
2    b2     UNKNOWN
3    b3     UNKNOWN
4    b4     UNKNOWN
5    b5     UNKNOWN
6    b6     UNKNOWN
7    b7     UNKNOWN
8  ndvi     UNKNOWN
9  savi     UNKNOWN
10  evi     UNKNOWN

Spatial extent: (, ) - (, )
Time range:  to 
Temporal resolution:  days 
Spatial resolution:  metres 
----------------------------------------------------------------------------------
Coverage: merge
Description: UNKNOWN
Source: UNKNOWN
Bands: 
  name description
1 prec     UNKNOWN
2 nest     UNKNOWN

Spatial extent: (, ) - (, )
Time range:  to 
Temporal resolution:  days 
Spatial resolution:  metres 
----------------------------------------------------------------------------------
\end{Soutput}
\begin{Sinput}
> # then, configure the WTSS service
> inpe <- sits_configWTSS (URL,
+                  coverage = "chronos:modis:mod13q1_512",
+                  bands = c("ndvi", "evi", "nir"))
> # a complicated point
> long <- -55.51810
> lat <-  -11.63884
> series.tb <- sits_getdata(longitude = long, latitude = lat, wtss = inpe)